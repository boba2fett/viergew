	\documentclass[12pt,a4paper,ngerman]{article}
	
	\usepackage[utf8]{inputenc}
	\usepackage[T1]{fontenc}
	\usepackage{babel}
	\usepackage{graphicx} 
	\usepackage{hyperref} 
	%\usepackage[german]{algorithm2e}
	%\glqq \grqq{}
	\title{Vier-Gewinnt}
	\author{}
	
	\begin{document}
	\pagenumbering{gobble}
	\maketitle
	\newpage
	\tableofcontents
	\newpage
	\pagenumbering{arabic}
	\section{Einleitung}
	
	\section{Vier Gewinnt}
	\subsection{Spielidee}
	Die Grundidee des Vier Gewinnt Spieles ist es vier der eigenen Spielsteine bzw. Symbole in eine Reihe zu bringen.
	Dabei ist es egal ob dies horizontal, vertikal oder diagonal erreicht wird.
	Es spielen immer zwei Spieler gegeneinander, die abwechselnd einen Spielstein von oben in das Spielfeld einfügen, der dann so lange herunterfällt, bis er auf das Ende des Spielfeldes oder einen anderen Spielstein stößt.
	Sobald zum ersten mal vier Spielsteine eines Spielers eine Reihe bilden, endet das Spiel und der betreffende Spieler gewinnt.
	Das Spielfeld in der Grundversion ist ein 7x6 Feld. Grundsätzlich sind aber auch größere und kleinere Felder möglich.
	\footnote{Lehmann, Jörg \glqq Vier gewinnt\grqq{} \url{http://www.brettspiele-report.de/vier-gewinnt/} Stand: 01.05.2019}
	\subsection{Implementation}
	\textit{klassendiagramm}
	\section{Implementation der KI}
	\subsection{Taktik der KI}
	Die Taktik der KI ist natürlich zu Gewinnen oder eine (unmittelbare) Niederlage zu verhindern. Dazu wird ein Minimax-Algorithmus genutzt, um die beste Entscheidung zu treffen. Bei diesem Verfahren durchläuft man einen Suchbaum, der bei der maximalen Tiefe jedem Ergebnis einen Wert zuweist oder wenn schon vorher ein Sieg oder eine Niederlage auftritt diesen jeweils einen positiven bzw. negativen Wert zuordnet. Bei der Auswertung wird dann immer das Minimum der gegnerischen Züge gewertet und das Maximum der eigenen, so dass jeder Spieler ein \glqq perfektes \grqq{} Spiel spielt. Des Weiteren bevorzugt der Algorithmus einen schnellen Sieg bzw. eine spätere Niederlagen.
		\\
	Neben dem Minimax-Algorithmus könnte man auch vorberechnete Eröffnungszüge nutzen, aber diese unterscheiden sich für jede Spielfeldgröße. Einige Spiele gelten dadurch auch schon als gelöst:
	\begin{figure}
		\centering
		\includegraphics[width=0.7\linewidth]{w-h-viergew}
		\caption{}
		\label{fig:w-h-viergew}
		\footnote{Tromp, John  \glqq John's Connect Four Playground \grqq{} \url{http://tromp.github.io/c4/c4.html} Stand: 01.05.2019}
	\end{figure}
    \newpage
	Zu Abbildung 1:\\
	+ der 1. Spieler gewinnt\\
	- der 2. Spieler gewinnt\\
	= Unentschieden\\
	
	
	
	\subsection{Schwächen}
	Theoretisch sind der KI nur Grenzen gesetzt durch die Rechenleistung bzw. die Tiefe der Simulation, aber um mehr Züge zu simulieren bräuchte man erheblich mehr Rechenleistung. Die Komplexität steigt nahezu exponentiell, aber je weiter das Spiel fortgeschritten ist und je weniger Möglichkeiten es gibt, desto schneller wird der Algorithmus. Ich habe Simulationen durchgeführt mit verschiedenen Feldgrößen und habe den ersten Zug von der KI berechnen lassen, dabei habe ich die Zeit gemessen und damit passende Tiefen gewählt, um die Zeit bei weniger als 10 Sekunden zu halten. Hätte man die vordefinierten Pfade genommen hätte man natürlich deutlich an Rechenzeit gespart, jedoch bräuchte man zum erstellen dieser sehr lange bzw. würde es den Rahmen dieser Facharbeit sprengen.
	\section{Künstliche Intelligenz}
	\textit{Definition}
	\subsection{Aktueller Stand}
	\textit{tensorflow,machine learning}
	\subsection{Ausblick}
	\textit{Filme usw. (terminator,2001,I Robot)}
	
\end{document}

