	\documentclass[12pt,a4paper,ngerman]{article}
	
	\usepackage[utf8]{inputenc}
	\usepackage[T1]{fontenc}
	\usepackage{babel}
	\usepackage{graphicx} 
	
	\title{Vier-Gewinnt}
	\author{}
	
	\begin{document}
	\maketitle
	%\newpage
	\tableofcontents
	\newpage
	\section{Vier Gewinnt}
	\subsection{Spielidee}
	Die Grundidee des Vier Gewinnt Spieles ist es vier der eigenen Spielsteine bzw. Symbole in eine Reihe zu bringen.
	Dabei ist es egal ob dies horizontal, vertikal oder diagonal erreicht wird.
	Es spielen immer zwei Spieler gegeneinander, die abwechselnd einen Spielstein von oben in das Spielfeld einfügen, der dann so lange herunterfällt, bis er auf das Ende des Spielfeldes oder einen anderen Spielstein stößt.
	Sobald zum ersten mal vier Spielsteine eines Spielers eine Reihe bilden, endet das Spiel und der betreffende Spieler gewinnt.
	Das Spielfeld in der Grundversion ist ein 7x6 Feld, das aber schon als gelöst gilt (dazu später mehr).
	Grundsätzlich sind aber auch größere Felder möglich, die aber meistens nie über 8x8 hinausgehen.
	\subsection{Implementation}
	\textit{klassendiagramm}
	\section{Implementation der KI}
	\subsection{Taktik des NPCs}
	Die Taktik des NPCs ist natürlich zu Gewinnen oder eine (unmittelbare) Niederlage zu verhindern.
	\textit{MINIMAX methode}
	\subsection{Schwächen}
	\section{Künstliche Intelligenz}
	\textit{Definition}
	\subsection{Aktueller Stand}
	\textit{tensorflow,machine learning}
	\subsection{Ausblick}
	\textit{Filme usw. (terminator,2001,I Robot)}
	
\end{document}

